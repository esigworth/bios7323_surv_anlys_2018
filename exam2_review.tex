\documentclass[11pt]{article}
\usepackage{amsfonts, amssymb, amsmath, amsthm, graphicx, wasysym}
\DeclareGraphicsRule{*}{eps}{*}{}

\special{papersize=8.5in,11in}
\setlength{\pdfpageheight}{\paperheight}
\setlength{\pdfpagewidth}{\paperwidth}

\setlength{\textwidth}{7.5in} \setlength{\textheight}{9.5in}
\setlength{\oddsidemargin}{-.5in} \setlength{\parskip}{0.15in}
\setlength{\topmargin}{-0.5in} \pagestyle{empty}
\setlength{\headheight}{0in} \setlength{\headsep}{0in}
\setlength{\parindent}{0in}

\allowdisplaybreaks

\begin{document}

\underline{BIOS 7323 - Exam 2 Review}

\textbf{Chapters 1-4, 7}
\begin{itemize}
	\item Relationships between pdf, survival function, hazard function, cumulative hazard function
	\item Recognize commonly used distributions (Exponential, Weibull, Gamma)
	\item Non-parametric estimates of basic quantities, esp. Kaplan-Meier survival estimates
	\item Hypothesis tests, esp. log-rank test
\end{itemize}

\textbf{Chapter 8 - Semi-parametric Proportional Hazards (PH) models}
\begin{itemize}
	\item Know form of PH model:	$h(t|Z)=h_0(t)C(\beta'Z)$, usually $C(t)=\exp(t)$
	\item Recognize and be able to calculate the partial likelihood for distinct event time data
	
	$\mathcal{PL}=\prod_{i=1}^{D}\frac{\exp(\beta'Z_{(i)})}{\sum_{j\in R(t_i)} \exp(\beta'Z_{(j)})}$, $D$ are uniq. death times\\
	$R(t_i)=\{j:1\le j \le n,T_j \ge t_i\}$ ind. in study just prior to $t_i$
	
	Note: PH model uses ranks and censoring, not actual times
	
	\item Know the three methods for dealing with ties (Breslow, Efron, and Cox) and how they differ
	
	with ties $d_i\ge 1$ at each $t_i, i=1,\ldots,D$, $D(t_i)$=set of ind. who die at $t_i$
	
	Breslow - Use naive PL, assume no ties. good w/ few ties
	
	$\mathcal{PL}_1(\beta)=
	\prod_{i=1}^{D}\frac{\exp(\beta'Z_{(i)})}{[\sum_{j\in R(t_i)} \exp(\beta'Z_{(j)})]^{d_i}}$
	
	Efron - Based on discrete hazard model. Closer to correct PL than Breslow. Breslow \& Efron similar with small \# ties.
	
	$\mathcal{PL}_2(\beta)=
	\prod_{i=1}^{D}\frac{\exp(\beta'S_{i})}{
		\prod_{j=1}^{d_i}[\sum_{k\in R(t_i)} \exp(\beta'Z_{k}) - \frac{j-1}{d_i} \sum_{k\in D(t_i)} \exp(\beta'Z_{k}) ]} $
	
	Cox - Exact, but complicated \& computationally intensive. $Q_i$ is set of all subsets of the $d_i$ individuals who could be selected from risk set. $q=\{q_1,\ldots,q_{d_i}\}$ and $S_q^*=\sum_{j=1}^{d_i}Z_{qj}$
	
	$\mathcal{PL}_3(\beta)=
	\frac{\exp(\beta'S_i)}{\sum_{q \in Q_i}\exp(\beta'S_q^*)}$
	
	\item Three tests for PH regression model parameters (Wald, Partial LR or Score Test). Score test with one binary covariate is the same as log-rank.
	
	\item PH Regression model building
	
	Possible Criteria: Wald test, LR test, score test, AIC
	
	AIC$= -2\log \mathcal{L} + kp$, $k$ is penalty, $p$ is number of parameters
	
	\item Estimation of Survivor function
	
	$
	W(t_i,\hat{\beta})=\sum_{j \in R(t_i)}e^{\hat{\beta}'Z_j}\\
	\hat{H}_0(t)=\sum_{t_i \le t}\frac{d_i}{W(t_i,\hat{\beta})} \text{ (Breslow cum. hazard est.)}\\
	\hat{S}_0(t)=\exp(-\hat{H}_0(t)) \text{ (Baseline survival function)}\\
	\hat{S}(t|Z=Z_0)=[\hat{S}_0(t)]^{\exp(\hat{\beta}'Z_0)}
	$	
\end{itemize}


\textbf{Chapter 9 - Refinements of Semi-parametric PH models}
\begin{itemize}
 \item	Know form of PH model with time-dependent covariates: $h(x|Z(t),t\le x)=h_0(x)\exp(\beta'Z(x))$ and how to interpret 
 
 \item Recognize and interpret \verb|R coxph| models using \verb|tt()| or counting process (start, end] intervals
 
 \item Approaches to deal with non-proportional hazards 1. piecewise PH model w/ TD vars 2. stratified models
 
 \item Know form of stratified PH model: $h_j(x|Z(t),t\le x)=h_{0j}(x)\exp(\beta'Z(x))$
 
 LR Test for assumption of common $\beta$ across $j$ strata:\\
 $\chi_{(s-1)p}^2 = 2[\sum_{j=1}^{s}LL_j(\hat{\beta}_{j}) - \sum_{j=1}^{s}LL_j(\hat{\beta})]$\\
 1st term from ind. models for each strata, 2nd term from stratified model 
 

 \item PH regression with left-truncation - condition hazard on $X>L$, modify risk set $R(t)=\{j:L_j<t\le T_j\}$.
 In \verb|R| use \verb|Surv(entry,failtime,status)| syntax
 
\end{itemize}


\textbf{Chapter 11 - Regression Diagnostics}
\begin{itemize}
	\item Overall Fit
	
	1. Cox-Snell residuals $r_j=\hat{H}_0(T_j)\exp(\hat{\beta}'Z_j)$\\
	2. Plot $\hat{H}_r(r_j)$ (cum. hazard based on $\{r_j, \delta_j\}$) vs. $r_j$. line through origin w/ slope 1 if good fit
	
	in \verb|R|, 
	\verb|cs_res<-delta-resid(fit,type="martingale")|
	
	\item Functional form of covariates
	
	1. Get martingale residuals (diff. between obs and exp deaths in (0,$t_i$)) $\hat{M}_j=\delta_j - \hat{H}_0(T_j)\exp(\hat{\beta}'Z_j)$ (for RC and time ind. var) from model where form of $Z_1$ is not known\\
	2. Scatterplot of $\hat{M}_j$ vs. $Z_1$ for $j$th obs. \& apply smoother\\
	3. smoothed curve suggests form for $f(Z_1)$
	
	in \verb|R|, 
	\verb|mg_res<-resid(fit,type="martingale")|
	
	\item PH assumption
	
	Approach 1 - Use time dependent covariate. 1. Multiply fixed covar by function of time $g(t)$ to create TD covar 2. fit PH model with fixed and TD covar; significant TD indicates PH violation
	
	Approach 2 - Cumulative Hazard plots. Discretize $Z_1$ into $K$ groups and fit models stratified on $Z_1$, $\log\{\hat{H}_{g0}(t)\}$ for $g=1,\ldots,K$\\
	$ \log\{\hat{H}_{g0}(t)\}$ vs. $t$ should be parallel\\
	$ \log\{\hat{H}_{g0}(t)\} - \log\{\hat{H}_{10}(t)\}$ vs. $t$ for $g=2,\ldots,K$ should be roughly constant\\
	$ \hat{H}_{g0}(t)$ vs. $\hat{H}_{10}(t)$ for $g=2,\ldots,K$ should be straight lines through origin (Andersen plot)
	
	Approach 3 - Arjas plot for categorical covar $Z_1$
	
	Approach 4 - Score residuals plot
	define process $U_k(t)$ for each covar. Plot of $U_k(t)$ vs. $t$ should fluctuate around 0 if PH holds. (within $\pm$ 1.358 - prob from Brownian bridge)
	
	in \verb|R|, \verb|sch_res<-resid(fit,type="schoenfeld")|\\ \verb|stdsc_res<-cumsum(sch_res)*sqrt(fit$var)|
	
	\item Outliers
	
	Deviance residuals less skewed than martingale residuals. Plot risk score vs. deviance resid. Large vals of deviance resid are outliers
	
	in \verb|R|, \verb|dev_res<-resid(fit1,type="deviance")|
	
	\item Influential points
	
	$\hat{\beta} - \hat{\beta}_{(j)}$ vs. $j$ where $\hat{\beta}_{(j)}$ is model w/o $j$. approximate using score residuals $I(\hat{\beta})^{-1}(S_{j1},\ldots,S_{jp})'$
	
	in \verb|R|, \verb|diff_betas<-resid(fit1,type="dfbetas")|
	
\end{itemize}


\textbf{Chapter 12 - Parametric Regression models}
\begin{itemize}
	\item Accelerated Failure time representation
	
	$S(x|Z)=S_{0}[\exp(\theta'Z)x]$, where $\exp(\theta'Z)$ is accel. factor
	
	$X_{0.5}^{(Z)}=\frac{X_{0.5}^{(0)}}{\exp(\theta'Z)}$
	
	\item Linear log time representation
	
	$Y=\log X = \mu + \gamma'Z + \sigma W$, where $W$ is known dist.
	
	If $S_0(x)$ is survival function of $\exp(\mu + \sigma W)$ then linear log time model $\Leftrightarrow$ AFT model with $\theta = -\gamma$.
	
	\textsl{Weibull}: $W$ is standard extreme value distribution. Has linear log time, AFT, and PH representations
	
	$h(x|Z)=\alpha \lambda x^{\alpha - 1}\exp(\beta'Z)$
	
	Convert between linear log time and hazard parameters:\\
	$\alpha=1/\sigma \quad \lambda=\exp(-\mu/\sigma) \quad \beta_j=-\gamma_j/\sigma,\, j=1,\ldots,p$
	
	in \verb|R|, have $\log(\hat{\sigma})$ convert from $Cov(\hat{\mu},\log(\hat{\sigma}))$ to $Cov(\hat{\mu},\hat{\sigma})$:
	
	$ Cov(\hat{\mu},\hat{\sigma})= Cov(\hat{\mu}, \log(\hat{\sigma}))\hat{\sigma} \quad 
	Var(\hat{\sigma})=Var(\log\hat{\sigma})\hat{\sigma}^2$
	
	\textsl{Log-logistic}: $W$ is standard logistic distribution. Has linear log time, AFT, and prop. odds representations
	
	$S(x|Z)=\frac{1}{1+\lambda e^{\beta'Z}x^{\alpha} }$
	
	$\frac{S(x|Z)}{1-S(x|Z)}=\exp(-\beta'Z)
	\frac{S(x|Z=0)}{1-S(x|Z=0)}$
	
	Same parameter conversion from linear log time as Weibull	
\end{itemize}


\textbf{Sample Size and Study Design} 

\begin{itemize}
	\item Know steps to calculate sample size :
	
\item Crude estimate based on survival at fixed point:

$N_{arm}=\frac{\left(z_{1-\alpha/2}\sqrt{2\bar{P}(1-\bar{P})} +z_{1-\beta}\sqrt{P_e(1-P_e) + P_c(1-P_c)} \right)^2}{(P_c-P_e)^2}$

$P_c$: prob of event in control arm by time $t$\\
$P_e$: prob of event in "experimental" arm by time $t$\\
$\bar{P}=(P_e + P_c)/2$

\item Sample size based on log-rank test:

HR: $\theta=e^{\beta}=\frac{\lambda_1(t)}{\lambda_0(t)}$

Number of events, $d$, needed for power $1-\beta$ with two-sided $\alpha$ level test is $d=\frac{4(z_{1-\alpha/2}+z_{1-\beta})^2}{[\log(\theta)]^2}$

Estimate $\theta$ from desired R-year survival in group 1, $S_{1}(R)$ and group 0, $S_{0}(R)$ (under exponential distribution)\\
$\frac{\log(S_{1}(R))}{\log(S_{0}(R))}=\frac{-\lambda_1 R}{-\lambda_0 R}=\frac{\lambda_1}{\lambda_0}=\theta$

Estimate $\theta$ from desired improvement in median survival from $M_0$ months to $M_1$ months  (under exponential distribution)\\
$\lambda_{i}=\frac{-\log(0.5)}{M_{i}},\, i=0,1$ 

How many patients? for follow-up time $F$,\\
$d=(N/2)(1-e^{-\lambda_0 F}) + (N/2)(1-e^{-\lambda_1 F})$

\item More realistic accrual (not all entries on same day) for accrual period, $A$.

to get $P_c$ and $P_e$ solve $P_i=1-\frac{\exp(-\lambda_i F)(1-\exp(-\lambda_i A))}{\lambda_i A}$
or
$P_i \approx 1- \exp[-\lambda_i(A/2 + F)]$ where $i=c,e$

then
$N = \frac{2d}{P_c + P_e} \quad N=\frac{8(z_{1-\alpha/2}+z_{1-\beta})^2}{[\log(\theta)]^2(P_c+P_e)}$

Vary $A$ and $F$ to find study design that has large enough sample and is feasible given expected accrual

Freedman approx. (conservative)\\
$N=\frac{2(z_{1-\alpha/2} + z_{1-\beta})^2}{P_e + P_c}\left(\frac{\theta + 1}{\theta -1}\right)^2$

\end{itemize}







$\square$
\end{document} 